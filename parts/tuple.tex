\section{Tuple (Demet)}

Tuple sıralı, değiştirilemez (immutable) bir koleksiyon oluşturur. Tuple, birden fazla veri öğesini bir arada saklamak için kullanılır. Tuple, elemanların sırasını korur. Yani, öğelere indeks numaraları ile erişilebilir. Tuple oluşturulduktan sonra, içindeki öğeler değiştirilemez, eklenemez veya silinemez.Tuple, birden fazla veri türünü bir arada saklayabilir. Tuple'lar içinde başka tuple'lar saklanabilir. Tuple'lar, listelere göre daha az bellek alanı kaplar. Çünkü değiştirilemez olmaları dolayısıyla bellekte daha az yönetim gerektirir.

Tuple'lar, değiştirilemez bir yapıda oldukları için bellekte daha az yer kaplarlar ve daha hızlı çalışırlar. Bellekte sabit bir konumda tutulurlar ve bu da erişim sürelerini kısaltır. Tuple oluşturulurken, bellekte bir dizi sürekli alan (contiguous memory) tahsis edilir. Bu sayede tuple, veri okuma ve yazma işlemlerinde daha hızlıdır.

\subsection{Erişim (Access)}

Tuple içindeki öğelere erişim, indeks numarası kullanılarak yapılır. Tuple'daki öğeler 0'dan başlayarak indekslenir. Zaman karmaşıklığı $O(1)$ çünkü belirli bir indeksteki öğeye doğrudan erişilebilir.

\subsubsection{Python - Indexing}

\begin{lstlisting}[language=Python]
my_tuple[1]
\end{lstlisting}

\subsection{Arama (Search)}

Tuple içinde belirli bir öğenin var olup olmadığını kontrol etmek için tüm elemanlar kontrol edilir. En kötü durumda tüm öğeleri kontrol etmek gerekebilir. Zaman karmaşıklığı $O(n)$'dir.

\subsubsection{Python - Search}

\begin{lstlisting}[language=Python]
for i in my_tuple:
    if i == search:
        return my_tuple.index(i)
\end{lstlisting}

\subsection{Ekleme (Insertion)}

Tuple'lar değiştirilemez olduğu için doğrudan ekleme işlemi yapılamaz. Bununla birlikte, mevcut bir tuple ile yeni bir tuple birleştirerek yeni bir tuple oluşturulabilir. Bu, temelde yeni bir tuple yaratmak anlamına gelir. Yeni bir tuple oluşturulurken mevcut tuple'daki tüm öğelerin kopyalanması gerekir. Zaman karmaşıklığı $O(n)$'dir.

\subsubsection{Python - Add}

\begin{lstlisting}[language=Python]
new_tuple = my_tuple + (40, 50)
\end{lstlisting}

\subsection{Silme (Deletion)}

Tuple'lar da değiştirilemez olduğu için doğrudan silme işlemi yapılamaz. Ancak, bir tuple'dan belirli öğeleri çıkarmak için yine yeni bir tuple oluşturulabilir. Yeni bir tuple oluşturulurken mevcut tuple'daki tüm öğelerin gözden geçirilmesi gerekir. Zaman karmaşıklığı $O(n)$'dir.

\subsubsection{Python - Deletion}

\begin{lstlisting}[language=Python]
new_tuple = tuple(x for x in my_tuple if x != 20)
\end{lstlisting}

\newpage